\section{Approach}
\label{sec:approach}

This section presents our approach to learning interpretable driving rules from demonstrations. We begin by introducing the concept of condition-action pairs in Sec.~\ref{sec:condition-action-pair}, which forms the foundation of our rule representation. Sec.~\ref{sec:learnable_logic_structure} introduces the core of our method: the learnable logic structure. Here, we explain its components, analyze its capabilities, and describe how we extract and simplify rules from it. Finally, Sec.~\ref{sec:learning_from_single_class_demonstration} introduces our novel regularization techniques to overcome the limitations of learning from positive-only examples.

\subsection{Condition-Action Pair}
\label{sec:condition-action-pair}
We consider rules that consist of conditions and expected actions. For instance, one such rule might require that the ego car eventually stop when approaching a stop sign. This pattern of condition-action pairing extends to countless driving situations. Thus, we focus on effectively learning and reducing driving rules to condition-action pairs.

Predicates are the basic unit of our rules. We categorize our predicates into two types: condition predicates $\bar{\PredicateSet} = \{\bar{\PredicateWithIndex{1}}, \bar{\PredicateWithIndex{2}}, \ldots, \bar{\PredicateWithIndex{n}}\}$ and action predicates $\dot{\PredicateSet} = \{\dot{\PredicateWithIndex{1}}, \dot{\PredicateWithIndex{2}}, \ldots, \dot{\PredicateWithIndex{m}}\}$. Condition predicates evaluate traffic conditions (e.g., if approaching a stop sign), while action predicates assess the motion plan (e.g. if the ego car is stopped). Given
\begin{equation}
    \begin{aligned}
        \mathit{condition} & := \bar{\Predicate} \mid \G\ condition \mid \F\ \mathit{condition} \mid \lnot \mathit{condition} \\
        \mathit{action}    & := \dot{\Predicate} \mid \G\ \mathit{action} \mid \F\ \mathit{action} \mid \lnot \mathit{action}
    \end{aligned}
\end{equation}
we extract and simplify the learned logic formula $\mathcal{L}$ to a set of propositional rules of the form:
\begin{equation}
    \label{eq:condition_action_pair}
    \bigodot_{i=1}^m \left(\bigwedge_{j=1}^n condition_j \rightarrow action_i\right)
\end{equation}
where $\bigodot$ denotes that this condition-action pairs are connected by $\land$ or $\lor$ operators. The conjunction of conditions ($\bigwedge_{j=1}^n condition_j$) allows for more precise and specific criteria to be defined for each action, thereby describing precisely when the action should be allowed. Building upon this foundation, we introduce a learnable logic structure to represent and learn these condition-action pairs.


\subsection{Learnable Logic Structure}
\label{sec:learnable_logic_structure}

The learnable logic structure $\bar{\mathcal{L}}$ is a directed acyclic computation graph that represents a compositional logic formula. It consists of three types of layers: \textbf{Temporal, Propositional}, and \textbf{Aggregation}. These layers are interconnected through learnable gates that determine the flow and combination of logical operations. An example of this structure is shown in Fig.~\ref{fig:learnable_logic_structure}.

\begin{figure}[!htp]
    \centering
    % \vspace{-.5cm}
    \includegraphics[width=\linewidth]{imgs/temporal_logic_net.drawio.png}
    % \vspace{-.8cm}
    \caption{
        The logic structure $\bar{\mathcal{L}}$ consists of three types of layers: \textbf{Temporal, Propositional}, and \textbf{Aggregation}. The \textbf{Temporal} layer processes the initial predicates, applying temporal operators. The \textbf{Propositional} layer generates all possible pairs of predicates connected by logical operators. The \textbf{Aggregation} layer aggregates the output of the \textbf{Propositional} layer into one cluster by deciding the logic operator to connect neighboring clusters. \textbf{Temporal} layers can be stacked. a layer's formal definition is in Sec.~\ref{sec:layer_definition}.
        Two types of gates, the \selectiongate, and the \negationgate, are used to control the logic operators and the sign of the cluster inputs, respectively. Each clear circle ($\bigcirc$) in these gates represents a single value \emph{weight}. In the selection gate, the
        \protect\circled{graycirc} circle represents the operator with the largest weight, meaning the operator is selected. In the negation gate, the \protect\circled{bluecirc} circle represents the negation of the input (i.e., multiply with a negative number), while the \protect\circled{redcirc} circle represents the original input (i.e., a positive number). The gate implementation is described in Sec.~\ref{sec:gate_implementation}. Supposing we only consider one layer of Temporal layer ($n = 1$), and given a set of predicates $\PredicateSet = \{\PredicateWithIndex{1}, \PredicateWithIndex{2}, \PredicateWithIndex{3}\}$, $\PredicateWithIndex{2} \in \bar{\PredicateSet}$ and $\PredicateWithIndex{1}, \PredicateWithIndex{3} \in \dot{\PredicateSet}$, this learnable logic structure represents the logic formula $(\G \PredicateWithIndex{1} \lor \lnot \PredicateWithIndex{2}) \lor (\lnot \G \PredicateWithIndex{1} \land \F \PredicateWithIndex{3}) \lor (\lnot \PredicateWithIndex{2} \land \F \PredicateWithIndex{3})$
        This formula can be further reduced to $\PredicateWithIndex{2} \rightarrow (\G \PredicateWithIndex{1} \lor \F \PredicateWithIndex{3})$.
    }
    % \vspace{-.7cm}
    \label{fig:learnable_logic_structure}
\end{figure}

\subsubsection{Layer Definition}
\label{sec:layer_definition}
In the bottom block of Fig.~\ref{fig:learnable_logic_structure}, the frame batch is processed through $\PredicateSet$ and then passed through \textbf{Temporal} layers. Let $F = \{f_1, f_2, ..., f_T\}$ be a sequence of $T$ frames, where each frame represents a time point. Define $\PredicateSet = \{\PredicateWithIndex{1}, \PredicateWithIndex{2}, ..., \PredicateWithIndex{N}\}$ as a set of $N$ predicates. For each predicate $\PredicateWithIndex{i} \in \PredicateSet$ and each frame $f_t \in F$, we compute $X_i^t = \PredicateWithIndex{i}(f_t)$. Let $X_i^T = [x_i^1, x_i^2, ..., x_i^T]$ be the sequence of predicate values for predicate $\PredicateWithIndex{i}$ across all time steps. The output of the predicates is then defined as $\mathcal{X}^T = \{X_1^T, X_2^T, ..., X_N^T\}$, where each $X_i^T \in \mathbb{R}^T$ contains the predicate values computed over the entire time sequence for the $i$-th predicate. These frame sequences can be batched as shown in Fig.~\ref{fig:learnable_logic_structure}, to simplify the symbols, we only discuss the case with batch size one in the following parts.

The \textbf{Temporal} layer $\mathcal{T}$ operates on each element $X_i^T \in \mathcal{X}^T$, potentially applying a temporal operator. Formally, the output of $\mathcal{T}$ is $ O^T  = \mathcal{T}(\mathcal{X}^T) = \{o^T_1, o^T_2, \ldots, o^T_N\}$, where $o^T_i = \mathcal{T}(x_i^T) \in \{\mathbf{G} X_i^T, \mathbf{F} X_i^T, X_i^T\}$. Temporal layers can be stacked, allowing for the composition of temporal operators. For instance, with two stacked temporal layers, we could have $\mathcal{T}_2(\mathcal{T}_1(X_i^T)) = \mathbf{F}(\mathbf{G}(X_i^T))$. This composition allows for expressing more nuanced and complex temporal properties.

The \textbf{Propositional} logic layer $\mathcal{F}$ operates on its input set $O^T$, generating ${N \choose 2}$ clusters, each containing a combination of two inputs connected by a logical operator. The behavior of $\mathcal{F}$ is formalized as $\mathcal{F}(O^T) = O^P = \{o_1^P, o_2^P, \ldots, o_{N \choose 2}^P\}$, where $o_i^P = (\lnot) o_j^T \circ (\lnot) o_k^T$, $\circ \in \{\land, \lor\}$, and $O^T = \{o^T_1, o^T_2, \ldots, o^T_N\}$ is the output of a \textbf{Temporal} layer. Here, $j$ and $k$ represent the indices of the two different inputs being combined. Each input can be negated or unchanged when passing through a logic layer. We do not stack the \textbf{Propositional} layer as it would lead to exponential growth in the number of clusters; aggregating on one \textbf{Propositional} layer can represent any formula in the form of \eqref{eq:condition_action_pair} as we show in \ref{sec:logic_space_analysis}.

The \textbf{Aggregation} layer aggregates the output of the \textbf{Propositional} ($O^P$) layer into one cluster using logical connectives $\{\land, \lor\}$ to connect neighboring clusters. Formally, given the input $O^P$, the output of the \textbf{Aggregation} layer can be represented as $\mathcal{A}(O^P) = o^P_1 \circ_1 o^P_2 \circ_2 \ldots \circ_{{N \choose 2} - 1} o^P_{N \choose 2}$, where $\circ$ represents $\land$ or $\lor$.

A logic structure can be formally defined as
$
    \bar{\mathcal{L}}  = \mathcal{A}(\mathcal{F}(\mathcal{T}^{\times n}(\PredicateSet))),
$
where $n$ is the number of stacked Temporal layers.

\subsubsection{Gate Implementation}
\label{sec:gate_implementation}
The layers in Fig.~\ref{fig:learnable_logic_structure} are composed of selection gates and negation gates. Each $\bigcirc$ in these gates represents a weight $w \in \mathbb{R}$.

The \selectiongate~ acts as a soft attention mechanism to select between different operators across all layers (i.e., $\G, \F$ and identity operator in the Temporal layer, $\land, \lor$ in the \textbf{Propositional} and \textbf{Aggregation} layers), defined as:
\begin{equation}
    g_s(O) = \sigma([w_s^1 ; \cdots ; w_s^k]) \cdot [o_1 ; \cdots ; o_k]^T
\end{equation}
where $\sigma(\cdot)$ denotes the softmax function. For temporal operators in the \textbf{Temporal} layer, $O = [\rho(\G X_i^T), \rho(\F X_i^T), \rho(X_i^T)]^\top \in \mathbb{R}^3$ and $k=3$.
For logic operators in the \textbf{Propositional} and \textbf{Aggregation} layers, $O = [\rho(o_1 \land o_2), \rho(o_1 \lor o_2)]^\top \in \mathbb{R}^2$ and $k=2$. The evaluation function $\rho$ is defined in  \eqref{eq:temporal_operators} and \eqref{eq:fol_operators}. The selection gate blends operators using softmax to enable continuous gradient flow during training.

The \negationgate, given by $g_n = \tanh(w_{neg}) \cdot x$, with learnable parameters $w_{neg}$, controls the sign of cluster $x$ inputs to the Propositional layer.
\begin{equation}
    g_n = \tanh(w_{neg}) \cdot x, \quad w_{neg} \in \mathbb{R}, x \in \mathbb{R}
\end{equation}
The $\tanh$ function constrains the output to $[-1, 1]$. According to the quantitative semantics defined in \eqref{eq:fol_operators}, multiplying by a negative value negates the input. The gradient properties of $\tanh$ encourage $w_{neg}$ to converge towards either $-1$ or $1$ during training, effectively learning whether to negate the input.

\subsubsection{Logic Space Analysis}
\label{sec:logic_space_analysis}
Stacking \textbf{Temporal} layers monotonically increases the logic space. Creating a logic space that contains up to $n$ nested temporal operators can be easily achieved by stacking $n$ \textbf{Temporal} layers. For the \textbf{Propositional} layer, we assert that:

\begin{theorem}
    Aggregating the output from a single \textbf{Propositional} layer can represent any formula in the form of \eqref{eq:condition_action_pair}.
\end{theorem}

\begin{proof}
    Consider a \textbf{Propositional} layer with inputs ${\mathit{condition_1, \ldots, condition_n, action_1, \ldots, action_m}}$. For each $\mathit{action_i}$, combine:
    $$(\lnot \mathit{condition_1 \lor action_i) \land \ldots \land (\lnot condition_n \lor action_i})$$
    This is equivalent to $(\mathit{condition_1 \land \ldots \land condition_n) \rightarrow action_i}$. Aggregating for all $\mathit{action_i}$ yields:
    $$\bigodot_{i=1}^m \left((\mathit{condition_1 \land \ldots \land condition_n) \rightarrow action_i}\right)$$
    where $\bigodot$ is either $\land$ or $\lor$, matching the theorem's form.
\end{proof}

This theorem proves that a single \textbf{Propositional} layer is sufficient to express all condition-action pairs, justifying our computationally efficient design choice.


\subsubsection{Ensembling}
\label{sec:ensembling}

Given a set of logic structures $\{\bar{\mathcal{L}}_1, \bar{\mathcal{L}}_2, \ldots, \bar{\mathcal{L}}_k\}$, we can ensemble them by aggregating their outputs with an additional \textbf{Aggregation} layer. Formally,
\begin{equation}
    \begin{aligned}
        \bar{\mathcal{L}}_{\text{ensemble}} & = \mathcal{A}(\bar{\mathcal{L}}_1, \bar{\mathcal{L}}_2, \ldots, \bar{\mathcal{L}}_k)
    \end{aligned}
\end{equation}
In practice, we noticed that ensembling multiple logic structures can improve the robustness of the learned rules.


\subsubsection{Rule Extraction and Simplification}
\label{sec:rule_extraction_and_simplification}

The rule extraction and simplification process transforms the learned logic structure into interpretable condition-action pairs. To interpret the learned logic structure $\bar{\mathcal{L}}$, we extract its logic formula $\mathcal{L}$ by traversing the structure and selecting the most probable operators by concretizing the selection gates and negation gates. Given a selection gate $g_s$, let $\mathcal{C}(\cdot)$ denote the concretizing function:
\begin{equation}
    \begin{aligned}
        \mathcal{C}(g_s) & = \argmax_{op} w^{op}_s, \quad op \in \begin{cases}
                                                                     \{\mathbf{G}, \mathbf{F}, \text{id}\}, \\
                                                                     \{\land, \lor\},
                                                                 \end{cases}
    \end{aligned}
\end{equation}
\label{eq:selection_gate_concretizing}
where $w^{op}_s$ represents the weights associated with each operator in the selection gate.
For the negation gate $g_n$, the concretizing is determined by the sign of the weight:
\begin{equation}
    \begin{aligned}
        \mathcal{C}(g_n) & = \begin{cases}
                                 \text{negation}, & \text{if } w_{neg} < 0 \\
                                 \text{original}, & \text{otherwise}
                             \end{cases}
    \end{aligned}
\end{equation}
\label{eq:negation_gate_concretizing}

A concrete example is shown in Fig.~\ref{fig:learnable_logic_structure} by iteratively applying \eqref{eq:selection_gate_concretizing} and \eqref{eq:negation_gate_concretizing} to the selection and negation gates. For an ensemble logic structure, we only need to apply the same rule further for the additional Aggregation layer.

The extracted formula is then simplified to a set of condition-action pairs in the form of \eqref{eq:condition_action_pair}.
We apply the Quine-McCluskey algorithm~\cite{McCluskey1956MinimizationOB} to simplify the extracted formula.
Such simplification removes redundant cluster (e.g., $\land (\PredicateWithIndex{1} \lor \lnot \PredicateWithIndex{1})$).
The resulting formula is then converted to conjunctive normal form:
$
    \bigwedge_{i=1}^n \left( \bigvee_{j=1}^m \bar{P}_j \lor \bigvee_{k=1}^l \dot{P}_k \right),
$
where $\bar{P}_j$ and $\dot{P}_k$ represent condition and action predicates, respectively. By double negating the condition predicates and applying De Morgan's laws, this formula can be further simplified to condition-action pairs in the form of \eqref{eq:condition_action_pair}.


\subsection{Learning From Demonstration}
\label{sec:learning_from_single_class_demonstration}

Most demonstration datasets consist solely of examples representing ideal driving behaviors. Our goal is to learn the logic structure $\bar{\mathcal{L}}^*$ \footnote{$\bar{\mathcal{L}}$ means the learnable logic structure, while $\mathcal{L}$ means a concrete logic formula defined by \eqref{eq:logic_formula_syntax}. } and predicate parameters $\boldsymbol{\theta}^*$ that grade demonstrations with the highest score and unseen behaviors with lower scores. Directly optimizing on \eqref{eq:objective} would simply make the score $\bar{\mathcal{L}}(S; \boldsymbol{\theta})$ to be 1 for all demonstrations.
The optimizer could find ``shortcuts'' and result in learning trivial formulas like $\PredicateWithIndex{1} \lor \lnot \PredicateWithIndex{1}$, which are always true and do not provide meaningful rules.
To address this issue, we introduce two regularization terms for $\boldsymbol{\theta}$ and $\bar{\mathcal{L}}$. The full learning algorithm is presented in Algorithm \ref{algo:train-with-regularization}.
% \setlength{\textfloatsep}{0pt}
\begin{algorithm}[ht]
    \caption{Training with Regularization}
    \label{algo:train-with-regularization}
    \KwIn{Dataset $\mathcal{D}^+$, initial logic structure $\bar{\mathcal{L}}$, initial parameters $\boldsymbol{\theta}$, learning rates $\alpha$, $\beta$, max weight $w_{\max}$, batch size $B$, number of epochs $E$}
    \KwOut{Optimized $\bar{\mathcal{L}}^*$ and $\boldsymbol{\theta}^*$}

    \SetKwFunction{FUpdate}{Update}
    \SetKwFunction{FRegularize}{Regularize}

    \SetKwProg{Fn}{Function}{:}{}
    \Fn{\FUpdate{$\bar{\mathcal{L}}, \boldsymbol{\theta}, \mathcal{J}, \gamma$}}{
        Update $\boldsymbol{\theta}$ with $\nabla_{\boldsymbol{\theta}} \mathcal{J}$\;
        Update $\bar{\mathcal{L}}$ (i.e., gate weights) with $\nabla_{\bar{\mathcal{L}}} \mathcal{J}$\;
        \Return $\bar{\mathcal{L}}, \boldsymbol{\theta}$\;
    }

    \Fn{\FRegularize{$\bar{\mathcal{L}}, \boldsymbol{\theta}, \alpha, \beta, w_{\max}$}}{
        $\boldsymbol{\theta} \leftarrow \boldsymbol{\theta} - \alpha \cdot \text{sign}(\frac{\partial \bar{\mathcal{L}}}{\partial \boldsymbol{\theta}})$\;
        \ForEach{Aggregation layer in $\bar{\mathcal{L}}$}{
            $w_{\land} \leftarrow \min(w_{\land} + \beta, w_{\max})$\;
        }
        \Return $\bar{\mathcal{L}}, \boldsymbol{\theta}$\;
    }

    \For{epoch $\leftarrow 1$ \KwTo $E$}{
        \For{each batch $\{S_1, \ldots, S_B\} \sim \mathcal{D}^+$}{
            $\mathcal{J} \leftarrow \frac{1}{B}\sum_{i=1}^B \bar{\mathcal{L}}(S_i; \boldsymbol{\theta})$\;
            $\bar{\mathcal{L}}, \boldsymbol{\theta} \leftarrow$ \FUpdate{$\bar{\mathcal{L}}, \boldsymbol{\theta}, \mathcal{J}, \gamma$}\;
            $\bar{\mathcal{L}}, \boldsymbol{\theta} \leftarrow$ \FRegularize{$\bar{\mathcal{L}}, \boldsymbol{\theta}, \alpha, \beta, w_{\max}$}\;
        }
    }
    \Return $\bar{\mathcal{L}}^* \leftarrow \bar{\mathcal{L}}$, $\boldsymbol{\theta}^* \leftarrow \boldsymbol{\theta}$
\end{algorithm}


The learning system evaluates driving behaviors in a state space $S = \{(\mathcal{E}_t, \tau_t)\}_{t=0}^T$, where behaviors with $\bar{\mathcal{L}}(S; \boldsymbol{\theta}) > 0$ are considered acceptable. Without constraints, the system might learn shortcuts that accept almost any behavior. We prevent this through two types of regularization:

\subsubsection{Preventing Shortcut Learning Through Regularization}

The learning system evaluates driving behaviors in a state space $S = \{(\mathcal{E}_t, \tau_t)\}_{t=0}^T$, where behaviors with $\bar{\mathcal{L}}(S; \boldsymbol{\theta}) > 0$ are considered acceptable. Without constraints, the system might learn shortcuts that accept almost any behavior.
Consider a simple example where we want to learn rules for safe lane changes. Without regularization, the system might learn the rule:
$\text{InLane} \lor \text{SafeDistance}$
that accepts behaviors in which the car is either in a lane OR maintains safe distance, which is clearly too permissive. We prevent such shortcuts through two types of regularization: First, we gradually tighten the parameters $\boldsymbol{\theta}$ of each rule using:
$
    \boldsymbol{\theta} = \boldsymbol{\theta} - \alpha \cdot \text{sign}\left(\frac{\partial \bar{\mathcal{L}}}{\partial \boldsymbol{\theta}}\right).
$
In our example, this might gradually increase the required safe distance threshold from 1 meter to a more reasonable 2 meters. Second, when combining rules, we encourage the use of AND ($\land$) over OR ($\lor$) operations using:
$
    w_{op}^{\land} = \min(w_{op}^{\land} + \beta, w_{\max})
$
This helps the system learn more appropriate rules like:
$\text{InLane} \land \text{SafeDistance}$
requiring both conditions to be met for a safe lane change.

Importantly, these constraints only eliminate spurious shortcuts - any rule structure or parameter that genuinely captures important driving behavior will remain intact, as it will be consistently reinforced by the demonstration data despite the regularization pressure.